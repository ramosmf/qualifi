\documentclass[a4paper,11pt]{report}


%%%%%%%%%%%%%%%%%%%%%%%%%%%%
% University of Sussex thesis template
%%%%%%%%%%%%%%%%%%%%%%%%%%%%
% Modification History
%
% Based on usthesis.cls by Jonathon Read
% http://www.cogs.susx.ac.uk/users/jlr24/latex.html
% Modified by Anthony Smith, Feb 2007
% Incorporated into single thesis.tex file, Anthony Smith, 30 June 2008
% Minor alterations to page numbering, AJS, 25 July 2008
% New alternative hyperref options for print version, AJS, 11 Sep 2008
% "DRAFT" on header, AJS, 12 Sep 2008
%%%%%%%%%%%%%%%%%%%%%%%%%%%%


%%%%%%%%%%%%%%%%%%%%%%%%%%%%
% LINE SPACING
\newcommand{\linespacing}{1.5}
\renewcommand{\baselinestretch}{\linespacing}
%%%%%%%%%%%%%%%%%%%%%%%%%%%%


%%%%%%%%%%%%%%%%%%%%%%%%%%%%
% BIBLIOGRAPHY STYLE
% \usepackage{natbib}
% \bibliographystyle{plain} for [1], [2] etc.
% \bibliographystyle{apalike}
%%%%%%%%%%%%%%%%%%%%%%%%%%%%


%%%%%%%%%%%%%%%%%%%%%%%%%%%%
% OTHER FORMATTING/LAYOUT DECLARATIONS
% Graphics
\usepackage{graphicx,color}
\usepackage{epstopdf}
\usepackage[brazil]{babel}
\usepackage[T1]{fontenc}
\usepackage{ae} 
\usepackage[ansinew]{inputenc} 
% Some packages used in my master's thesis
\usepackage{indentfirst}
% The left-hand-side should be 40mm.  The top and bottom margins should be
% 25mm deep.  The right hand margin should be 20mm.
\usepackage[a4paper,top=2.5cm,bottom=2.5cm,left=3cm,right=3cm,headsep=10pt]{geometry}
\flushbottom
% Pages should be numbered consecutively thorugh the main text.  Page numbers
% should be located centrally at the top of the page.
\usepackage{fancyhdr}
%\fancypagestyle{plain}{
	% \fancyhf{}
	% Add "DRAFT: <today's date>" to header (comment out the following to remove)
	% \lhead{\textit{DRAFT: \today}}
	%
	% \chead{\thepage}
	% \renewcommand{\headrulewidth}{0pt}
%}
\pagestyle{plain}
%%%%%%%%%%%%%%%%%%%%%%%%%%%%


%%%%%%%%%%%%%%%%%%%%%%%%%%%%
% ANY OTHER DECLARATIONS HERE:

%%%%%%%%%%%%%%%%%%%%%%%%%%%%


%%%%%%%%%%%%%%%%%%%%%%%%%%%%
% HYPERREF
\usepackage[colorlinks,pdfusetitle,urlcolor=blue,citecolor=blue,linkcolor=blue,bookmarksnumbered,plainpages=false]{hyperref}
% For print version, use this instead:
%\usepackage[pdfusetitle,bookmarksnumbered,plainpages=false]{hyperref}
%\usepackage{backref}
%\renewcommand{\backrefpagesname}{Cited on}
%%%%%%%%%%%%%%%%%%%%%%%%%%%%


%%%%%%%%%%%%%%%%%%%%%%%%%%%%
% BEGIN DOCUMENT
\begin{document}
%%%%%%%%%%%%%%%%%%%%%%%%%%%%


%%%%%%%%%%%%%%%%%%%%%%%%%%%%
% PREAMBLE: roman page numbering i, ii, iii, ...
\pagenumbering{roman}
%%%%%%%%%%%%%%%%%%%%%%%%%%%%


%%%%%%%%%%%%%%%%%%%%%%%%%%%%
%% TITLE PAGE: The title page should give the following information:
%%	(i) the full title of the thesis and the sub-title if any;
%%	(ii) the full name of the author;
%%	(iii) the qualification aimed for;
%%	(iv) the name of the University of Sussex;
%%	(v) the month and year of submission.
\thispagestyle{empty}
\begin{flushright}
\includegraphics[width=3cm]{puclogo}
\end{flushright}	
\vskip40mm
\begin{center}
% TITLE FENOMENOS DE PROPAGACAO EM REDES COMPLEXAS?

\huge\textbf{Exame de Qualifica\c{c}\~ao}
\vskip2mm
% SUBTITLE (optional)
\LARGE\textit{Fen\^omenos de Propaga\c{c}\~ao em Redes Complexas}
\vskip5mm
% AUTHOR
\Large\textbf{Aluno: Marlon Ramos}\\
\Large\textbf{Orientadora: Celia Anteneodo}
\normalsize
\end{center}
\vfill
\begin{flushleft}
\large
% QUALIFICATION
Departamento de F\'isica\\
Pontif\'icia Universidade Cat\'olica do Rio de Janeiro\\
% DATE OF SUBMISSION
Junho 2012
\end{flushleft}		
%%%%%%%%%%%%%%%%%%%%%%%%%%%%


%%%%%%%%%%%%%%%%%%%%%%%%%%%%
% SUMMARY PAGE
\thispagestyle{empty}
\newpage
\null\vskip10mm
\begin{center}
\large
\underline{Pontif\'icia Universidade Cat\'olica do Rio de Janeiro}
\vskip20mm
% AUTHOR, QUALIFICATION
\textsc{Marlon Ramos}
\vskip20mm
% TITLE
\underline{\textsc{Exame de Qualifica\c{c}\~ao}}
\vskip0mm
% SUBTITLE (optional)
\underline{\textsc{Fen\^omenos de Propaga\c{c}\~ao em Redes Complexas}}
\vskip20mm
\underline{\textsc{Resumo}}
\vskip2mm
\end{center}
% Change line spacing
\renewcommand{\baselinestretch}{1.0}
\small\normalsize
% SUMMARY HERE (300 word limit for most subjects):

%%%%%%%%%%%%%%%%%%%%%%%%%%%%


%%%%%%%%%%%%%%%%%%%%%%%%%%%%
% ACKNOWLEDGEMENTS
% \chapter*{Agradecimentos}
% \renewcommand{\baselinestretch}{\linespacing}
% \small\normalsize
% ACKNOWLEDGEMENTS HERE:

%%%%%%%%%%%%%%%%%%%%%%%%%%%%


%%%%%%%%%%%%%%%%%%%%%%%%%%%%
% TABLE OF CONTENTS, LISTS OF TABLES & FIGURES
\newpage
\pdfbookmark[0]{Contents}{contents_bookmark}
\tableofcontents
%\listoftables
%\phantomsection
%\addcontentsline{toc}{chapter}{List of Tables}
%\listoffigures
%\phantomsection
%\addcontentsline{toc}{chapter}{List of Figures}
%%%%%%%%%%%%%%%%%%%%%%%%%%%%


%%%%%%%%%%%%%%%%%%%%%%%%%%%%
% MAIN THESIS TEXT: arabic page numbering 1, 2, 3, ...
\newpage
\pagenumbering{arabic}
%%%%%%%%%%%%%%%%%%%%%%%%%%%%


%-----------------------------------------------------
% Chapter: Introducao e Metologia
%-----------------------------------------------------


\chapter{Introdu\c{c}\~ao e Metodologia}
\label{chap:intro}

\section{Introdu\c{c}\~ao e Motiva\c{c}\~ao}

CONTAGIO: (biologia e sociologia)  keywords PRx rumor+spreading

FOCO: fenomenos de propagacao em meios sociais: 

-difusao de rumores, informacoes

-surgimento de consenso na adopcao de inovacoes tecnologicas, 
 formacao de opinioes, da linguagem, diseminacao de ideias, 
da corrupcao, praticas sociais, etc.

-epidemias de doencas infecciosas 

diversos fenomenos governados por interacoes sociais
que podem evoluir para estados finais fragmentados desordenados 
ou de consenso, em que todos os individuos sao atingidos por 
informacoes ou doencas (consenso de opinioes, epidemias, 
respectivamente) ou adoptam um mesmo padrao comportamental 
(opiniao, linguagem) \cite{BOCCALETTI:2006gb}

a pergunta fundamental eh como esses estados estacionarios emergem, 
quais os mecanismos e como exercer controle sobre os mesmos


a topologia da rede em que as interacoes ocorrem tambem eh 
crucial para determinar as propriedades emergentes 

\section{Metologia} 

analogia com problemas de fisica estatistica, tr. de fase, percolacao, difusao, etc

fisica est: analogia com sistemas fisicos (magneticos tipo Ising, 1/2 ou 1)

uso de redes complexas

%-----------------------------------------------------
% Chapter: Modelos
%-----------------------------------------------------
\chapter{Revis\~ao Bibliogr\'afica}
\label{chap:models}

\section{Modelos}
-Individuos atingidos por rumores ou noticias <> analogo infecao doencas
(informados = infectados, porem: um eh intencional o outro involuntario, 
vantagem vs desvantagem) 

-as perguntas sao similares (numero de atingidos?, limiar de epidemia?, 
fracao finita vs entorno local), eg: viral strategies in marketting

- os objetivos sao opostos
- descrever em epidemias, inventar em informacoes

- interacao assimetrica
- fluxo unidireccional: vai de quem tem a info, doenca --> para quem nao tem 

-varios dos modelos supoem um limiar na fracao do numero de vizinhos "ativos"
porem em alguns casos um unico vizinho eh suficiente para iniciar o spreading

- Modelo standard DK (Daley Kendall) 
ignorants, spreaders, stifflers = S I R

I--> R
  k
  k(I+R)
\section{Redes Complexas}

\section{Resultados pr\'evios}


rumores: NO THRESHOLD FOR HOMOGENEOUS MIXING (for any rate a finite fraction will 
be informed)\cite{Castellano:2009ce}

Random, WS tem limiar epidemico, there is an epidemic transition \cite{Zanette:2001kh} 
correlacoes mudam o limiar
scale free NAO tem, epidemic threshold ou seja sempre tem epidemia

REDES em evolucao 
contact switching--> avoids spreading \cite{RisauGusman:2009jh}

%-----------------------------------------------------
% Chapter: Propostas e Conclus\~oes
%-----------------------------------------------------
\chapter{Propostas e Conclus\~oes}
\label{chap:conc}

- outros detalhes em modelos ja estudados
alem das propriedades estacionarias, eh importante caracterizar 
os transientes, quando leva em chegar no SS 

- novos

- difusao

- rumores --> generalizacoes


%%%%%%%%%%%%%%%%%%%%%%%%%%%%
% BIBLIOGRAPHY
%\newpage
\clearpage
\phantomsection
\addcontentsline{toc}{chapter}{Refer\^encias Bibliogr\'aficas}
\bibliography{bib}
\bibliographystyle{unsrt}
%%%%%%%%%%%%%%%%%%%%%%%%%%%%


%%%%%%%%%%%%%%%%%%%%%%%%%%%%
% START APPENDICES
% \appendix
%%%%%%%%%%%%%%%%%%%%%%%%%%%%


%-----------------------------------------------------
% Appendix: Code
%-----------------------------------------------------
%\chapter{Code}
%\label{app:code}

%Este phantom resolve o problema de que a Bibliografia n\~ao aparece no TOC
\phantom{x}

%\begin{verbatim}
%10 PRINT "HELLO WORLD"
%\end{verbatim}


%%%%%%%%%%%%%%%%%%%%%%%%%%%%
% END DOCUMENT
\end{document}
%%%%%%%%%%%%%%%%%%%%%%%%%%%%